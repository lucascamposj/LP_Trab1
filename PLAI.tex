\documentclass{book}

%% ODER: format ==         = "\mathrel{==}"
%% ODER: format /=         = "\neq "
%
%
\makeatletter
\@ifundefined{lhs2tex.lhs2tex.sty.read}%
  {\@namedef{lhs2tex.lhs2tex.sty.read}{}%
   \newcommand\SkipToFmtEnd{}%
   \newcommand\EndFmtInput{}%
   \long\def\SkipToFmtEnd#1\EndFmtInput{}%
  }\SkipToFmtEnd

\newcommand\ReadOnlyOnce[1]{\@ifundefined{#1}{\@namedef{#1}{}}\SkipToFmtEnd}
\usepackage{amstext}
\usepackage{amssymb}
\usepackage{stmaryrd}
\DeclareFontFamily{OT1}{cmtex}{}
\DeclareFontShape{OT1}{cmtex}{m}{n}
  {<5><6><7><8>cmtex8
   <9>cmtex9
   <10><10.95><12><14.4><17.28><20.74><24.88>cmtex10}{}
\DeclareFontShape{OT1}{cmtex}{m}{it}
  {<-> ssub * cmtt/m/it}{}
\newcommand{\texfamily}{\fontfamily{cmtex}\selectfont}
\DeclareFontShape{OT1}{cmtt}{bx}{n}
  {<5><6><7><8>cmtt8
   <9>cmbtt9
   <10><10.95><12><14.4><17.28><20.74><24.88>cmbtt10}{}
\DeclareFontShape{OT1}{cmtex}{bx}{n}
  {<-> ssub * cmtt/bx/n}{}
\newcommand{\tex}[1]{\text{\texfamily#1}}	% NEU

\newcommand{\Sp}{\hskip.33334em\relax}


\newcommand{\Conid}[1]{\mathit{#1}}
\newcommand{\Varid}[1]{\mathit{#1}}
\newcommand{\anonymous}{\kern0.06em \vbox{\hrule\@width.5em}}
\newcommand{\plus}{\mathbin{+\!\!\!+}}
\newcommand{\bind}{\mathbin{>\!\!\!>\mkern-6.7mu=}}
\newcommand{\rbind}{\mathbin{=\mkern-6.7mu<\!\!\!<}}% suggested by Neil Mitchell
\newcommand{\sequ}{\mathbin{>\!\!\!>}}
\renewcommand{\leq}{\leqslant}
\renewcommand{\geq}{\geqslant}
\usepackage{polytable}

%mathindent has to be defined
\@ifundefined{mathindent}%
  {\newdimen\mathindent\mathindent\leftmargini}%
  {}%

\def\resethooks{%
  \global\let\SaveRestoreHook\empty
  \global\let\ColumnHook\empty}
\newcommand*{\savecolumns}[1][default]%
  {\g@addto@macro\SaveRestoreHook{\savecolumns[#1]}}
\newcommand*{\restorecolumns}[1][default]%
  {\g@addto@macro\SaveRestoreHook{\restorecolumns[#1]}}
\newcommand*{\aligncolumn}[2]%
  {\g@addto@macro\ColumnHook{\column{#1}{#2}}}

\resethooks

\newcommand{\onelinecommentchars}{\quad-{}- }
\newcommand{\commentbeginchars}{\enskip\{-}
\newcommand{\commentendchars}{-\}\enskip}

\newcommand{\visiblecomments}{%
  \let\onelinecomment=\onelinecommentchars
  \let\commentbegin=\commentbeginchars
  \let\commentend=\commentendchars}

\newcommand{\invisiblecomments}{%
  \let\onelinecomment=\empty
  \let\commentbegin=\empty
  \let\commentend=\empty}

\visiblecomments

\newlength{\blanklineskip}
\setlength{\blanklineskip}{0.66084ex}

\newcommand{\hsindent}[1]{\quad}% default is fixed indentation
\let\hspre\empty
\let\hspost\empty
\newcommand{\NB}{\textbf{NB}}
\newcommand{\Todo}[1]{$\langle$\textbf{To do:}~#1$\rangle$}

\EndFmtInput
\makeatother
%
%
%
%
%
%
% This package provides two environments suitable to take the place
% of hscode, called "plainhscode" and "arrayhscode". 
%
% The plain environment surrounds each code block by vertical space,
% and it uses \abovedisplayskip and \belowdisplayskip to get spacing
% similar to formulas. Note that if these dimensions are changed,
% the spacing around displayed math formulas changes as well.
% All code is indented using \leftskip.
%
% Changed 19.08.2004 to reflect changes in colorcode. Should work with
% CodeGroup.sty.
%
\ReadOnlyOnce{polycode.fmt}%
\makeatletter

\newcommand{\hsnewpar}[1]%
  {{\parskip=0pt\parindent=0pt\par\vskip #1\noindent}}

% can be used, for instance, to redefine the code size, by setting the
% command to \small or something alike
\newcommand{\hscodestyle}{}

% The command \sethscode can be used to switch the code formatting
% behaviour by mapping the hscode environment in the subst directive
% to a new LaTeX environment.

\newcommand{\sethscode}[1]%
  {\expandafter\let\expandafter\hscode\csname #1\endcsname
   \expandafter\let\expandafter\endhscode\csname end#1\endcsname}

% "compatibility" mode restores the non-polycode.fmt layout.

\newenvironment{compathscode}%
  {\par\noindent
   \advance\leftskip\mathindent
   \hscodestyle
   \let\\=\@normalcr
   \let\hspre\(\let\hspost\)%
   \pboxed}%
  {\endpboxed\)%
   \par\noindent
   \ignorespacesafterend}

\newcommand{\compaths}{\sethscode{compathscode}}

% "plain" mode is the proposed default.
% It should now work with \centering.
% This required some changes. The old version
% is still available for reference as oldplainhscode.

\newenvironment{plainhscode}%
  {\hsnewpar\abovedisplayskip
   \advance\leftskip\mathindent
   \hscodestyle
   \let\hspre\(\let\hspost\)%
   \pboxed}%
  {\endpboxed%
   \hsnewpar\belowdisplayskip
   \ignorespacesafterend}

\newenvironment{oldplainhscode}%
  {\hsnewpar\abovedisplayskip
   \advance\leftskip\mathindent
   \hscodestyle
   \let\\=\@normalcr
   \(\pboxed}%
  {\endpboxed\)%
   \hsnewpar\belowdisplayskip
   \ignorespacesafterend}

% Here, we make plainhscode the default environment.

\newcommand{\plainhs}{\sethscode{plainhscode}}
\newcommand{\oldplainhs}{\sethscode{oldplainhscode}}
\plainhs

% The arrayhscode is like plain, but makes use of polytable's
% parray environment which disallows page breaks in code blocks.

\newenvironment{arrayhscode}%
  {\hsnewpar\abovedisplayskip
   \advance\leftskip\mathindent
   \hscodestyle
   \let\\=\@normalcr
   \(\parray}%
  {\endparray\)%
   \hsnewpar\belowdisplayskip
   \ignorespacesafterend}

\newcommand{\arrayhs}{\sethscode{arrayhscode}}

% The mathhscode environment also makes use of polytable's parray 
% environment. It is supposed to be used only inside math mode 
% (I used it to typeset the type rules in my thesis).

\newenvironment{mathhscode}%
  {\parray}{\endparray}

\newcommand{\mathhs}{\sethscode{mathhscode}}

% texths is similar to mathhs, but works in text mode.

\newenvironment{texthscode}%
  {\(\parray}{\endparray\)}

\newcommand{\texths}{\sethscode{texthscode}}

% The framed environment places code in a framed box.

\def\codeframewidth{\arrayrulewidth}
\RequirePackage{calc}

\newenvironment{framedhscode}%
  {\parskip=\abovedisplayskip\par\noindent
   \hscodestyle
   \arrayrulewidth=\codeframewidth
   \tabular{@{}|p{\linewidth-2\arraycolsep-2\arrayrulewidth-2pt}|@{}}%
   \hline\framedhslinecorrect\\{-1.5ex}%
   \let\endoflinesave=\\
   \let\\=\@normalcr
   \(\pboxed}%
  {\endpboxed\)%
   \framedhslinecorrect\endoflinesave{.5ex}\hline
   \endtabular
   \parskip=\belowdisplayskip\par\noindent
   \ignorespacesafterend}

\newcommand{\framedhslinecorrect}[2]%
  {#1[#2]}

\newcommand{\framedhs}{\sethscode{framedhscode}}

% The inlinehscode environment is an experimental environment
% that can be used to typeset displayed code inline.

\newenvironment{inlinehscode}%
  {\(\def\column##1##2{}%
   \let\>\undefined\let\<\undefined\let\\\undefined
   \newcommand\>[1][]{}\newcommand\<[1][]{}\newcommand\\[1][]{}%
   \def\fromto##1##2##3{##3}%
   \def\nextline{}}{\) }%

\newcommand{\inlinehs}{\sethscode{inlinehscode}}

% The joincode environment is a separate environment that
% can be used to surround and thereby connect multiple code
% blocks.

\newenvironment{joincode}%
  {\let\orighscode=\hscode
   \let\origendhscode=\endhscode
   \def\endhscode{\def\hscode{\endgroup\def\@currenvir{hscode}\\}\begingroup}
   %\let\SaveRestoreHook=\empty
   %\let\ColumnHook=\empty
   %\let\resethooks=\empty
   \orighscode\def\hscode{\endgroup\def\@currenvir{hscode}}}%
  {\origendhscode
   \global\let\hscode=\orighscode
   \global\let\endhscode=\origendhscode}%

\makeatother
\EndFmtInput
%

\usepackage{xspace}
\usepackage{tikz}
\usepackage{pgfplots}
\usepackage{exercise}
\usepackage{syntax} 

\pgfplotsset{compat=newest}
\usetikzlibrary{shapes.geometric,arrows,fit,matrix,positioning}
\tikzset
{
    treenode/.style = {circle, draw=black, align=center, minimum size=1cm},
    subtree/.style  = {isosceles triangle, draw=black, align=center, minimum height=0.5cm, minimum width=1cm, shape border rotate=90, anchor=north}
}


\newcounter{haskell}[chapter]
\newenvironment{haskell}[1][]{\refstepcounter{haskell}\par\medskip
   \noindent \textbf{Example~\thehaskell. #1} \rmfamily}{
\begin{hscode}\SaveRestoreHook
\ColumnHook
\end{hscode}\resethooks
}

\usepackage{amssymb,amsmath}

\usepackage{mdframed}
\usepackage{hyperref}

\global\mdfdefinestyle{default}{%
  linecolor=black,linewidth=0.5pt,
  backgroundcolor=gray!10
}

\usepackage[
    type={CC},
    modifier={by-nc-sa},
    version={4.0},
]{doclicense}

\newcommand{\bnf}{\texttt{BNF}}
\newcommand{\lae}{\textsc{LAE}\xspace}
\renewcommand{\ae}{\textsc{AE}\xspace}
\renewcommand{\emph}[1]{{\color{blue}\textit{#1}}}

\newcounter{mydef}[chapter] \setcounter{mydef}{0}
\renewcommand{\themydef}{\arabic{chapter}.\arabic{section}.\arabic{mydef}}
\newenvironment{mydef}[2][]{%
\refstepcounter{mydef}%
\ifstrempty{#1}%
{\mdfsetup{%
frametitle={%
\tikz[baseline=(current bounding box.east),outer sep=0pt]
\node[anchor=east,rectangle,fill=blue!20]
{\strut Definition~\themydef};}}
}%
{\mdfsetup{%
frametitle={%
\tikz[baseline=(current bounding box.east),outer sep=0pt]
\node[anchor=east,rectangle,fill=blue!20]
{\strut Definition~\themydef:~#1};}}%
}%
\mdfsetup{innertopmargin=10pt,linecolor=blue!20,%
linewidth=2pt,topline=true,%
frametitleaboveskip=\dimexpr-\ht\strutbox\relax
}
\begin{mdframed}[]\relax%
\label{#2}}{\end{mdframed}}


\title{A Short Introduction to Programming 
Languages: Application and Interpretation} 

\author{Rodrigo Bonif\'{a}cio, Luisa Fantin, Gabriel Lob\~{a}o, and Jo\~{a}o Sousa}

\begin{document}

\maketitle


\chapter*{Preface} 

This book presents a short introduction to 
Programming Languages: Application and 
Interpretation (the PLAI book from Shriram Krishnamurthi). 
Actually, it is a derivative of the aforementioned 
work, licensed under Creative Commons 
Attribution-NonCommercial-ShareAlike 3.0 
United States License. We selected several chapters 
from the original book, condensed them a bit, and 
migrated all Scheme source code to Haskell. 
Here our goal is to make an introduction to 
programming language operational 
semantics, which might help us to present 
related concepts to our Programming Languages 
students at University of 
Bras\'{i}lia. The original version of this work 
can be found at \url{www.plai.org/}. 

 

\doclicenseThis
\chapter{Interpreting Arithmetic and Substitution}

This chapter summarizes some relevants 
aspects of the first three chapters of the 
original version of this book, presents a 
small discussion about parsers and abstract syntax 
trees, evaluating simple arithmetic expressions, 
and one of the most fundamental aspects of the 
first part of this book: substitution. The reader 
should spend special dedication to this aspect, 
once it will be further explored in other chapters of this 
book. 

\section{A simple Arithmetic Expression Language} 

Having established a handle on parsing, 
which addresses syntax, we now begin to 
study semantics. We will study a language 
with only numbers, addition, and subtraction, 
and futher assume both these operations 
are binary. This is indeed a very rudimentary 
exercise, but that's the point. By picking 
something you know well, we can 
focus on the mechanics. Once you have a feel for the 
mechanics, we can use the same methods 
to explore languages you have nevever seen before. 

The interpreter has the following contract and purpose: 

\begin{hscode}\SaveRestoreHook
\column{B}{@{}>{\hspre}l<{\hspost}@{}}%
\column{50}{@{}>{\hspre}l<{\hspost}@{}}%
\column{E}{@{}>{\hspre}l<{\hspost}@{}}%
\>[B]{}\mathbf{module}\;\Conid{AE}\;\mathbf{where}{}\<[E]%
\\[\blanklineskip]%
\>[B]{}\mathbf{import}\;\Conid{\Conid{Test}.HUnit}{}\<[E]%
\\[\blanklineskip]%
\>[B]{}\mbox{\onelinecomment  consumes an AE and computes the corresponding number}{}\<[E]%
\\
\>[B]{}\Varid{calc}\mathbin{::}\Conid{AE}\to \Conid{Integer}{}\<[E]%
\\[\blanklineskip]%
\>[B]{}\mbox{\onelinecomment  some HUnit test cases to better understand the calc semantics}{}\<[E]%
\\[\blanklineskip]%
\>[B]{}\Varid{exp1},\Varid{exp2}\mathbin{::}\Conid{String}{}\<[E]%
\\
\>[B]{}\Varid{exp1}\mathrel{=}\text{\tt \char34 Num~3\char34}{}\<[E]%
\\
\>[B]{}\Varid{exp2}\mathrel{=}\text{\tt \char34 Add~(Num~3)~(Sub~(Num~10)~(Num~5))\char34}{}\<[E]%
\\[\blanklineskip]%
\>[B]{}\Varid{tc1}\mathrel{=}\Conid{TestCase}\;(\Varid{assertEqual}\;\text{\tt \char34 tc01\char34}\;(\Varid{calc}\;(\Varid{parse}\;{}\<[50]%
\>[50]{}\Varid{exp1}))\;\mathrm{3}){}\<[E]%
\\[\blanklineskip]%
\>[B]{}\Varid{tc2}\mathrel{=}\Conid{TestCase}\;(\Varid{assertEqual}\;\text{\tt \char34 tc02\char34}\;(\Varid{calc}\;(\Varid{parse}\;\Varid{exp2}))\;\mathrm{8}){}\<[E]%
\ColumnHook
\end{hscode}\resethooks

An arithmetic expression \texttt{AE} might be represented 
using a notation named \emph{Backus-Naur Form} (BNF), after
two early programming languages pioneers. A BNF 
description of rudimentary arithmetic looks like: 

\begin{tabbing}\tt
~~\char60{}AE\char62{}~\char58{}\char58{}\char61{}~Num~\char60{}int\char62{}~\\
\tt ~~~~~~~~~\char124{}~Add~\char60{}AE\char62{}~\char60{}AE\char62{}~\\
\tt ~~~~~~~~~\char124{}~Sub~\char60{}AE\char62{}~\char60{}AE\char62{}~
\end{tabbing}

The \texttt{<AE>} in the \textsc{BNF} is calleed a non-terminal, 
which means we can rewrite it as one of the things on 
the right-hand side. Read \texttt{::=} as ``can be rewritten as''. 
Each line presents one more choice, called a 
\emph{production}. Everything in a production that isn't enclosed 
in the symbos \texttt{< \ldots >} is literal syntax. In Haskell, 
as well as in other programming languages, it is quite easy to 
represent an abstract representation for arithmetic expressions 
based on a \textsc{BNF} specification. Abstract representations 
are independent of our choices to concretely represent arithmetic 
expressions (or other more advanced programming 
language constructs) as strings of characters. For instance, we might 
express an expression 3 + (10 - 5) in many different ways, for instance: 

\begin{tabbing}\tt
~\char43{}~3~\char40{}\char45{}~10~5\char41{}\\
\tt ~\char43{}~\char40{}3\char44{}~\char45{}\char40{}10\char44{}~5\char41{}\char41{}\\
\tt ~add\char40{}3\char44{}~sub\char40{}10\char44{}~5\char41{}\char41{}
\end{tabbing}

All these forms of expressing the same arithmetic expression 
could be \emph{parsed} to generate the same abstract 
syntax tree (a possible alternative is shown in Figure~\ref{ast:ae}). That is, 
after choosing an interesting concrete syntax for a language, 
a parser reads a program written according to the concrete 
syntax and ouputs an instance of an abstract syntax tree (AST). We will 
not give to much attention to concrete syntaxes and parsers in this 
book. Here, we are mostly intereted in the semantics of a programming 
language (in this case, expressed in terms of the \texttt{calc} 
function). We also have to define a \emph{data type} for representing the 
\texttt{AE} AST using Haskell (see Listing~\ref{}). Note 
how similar with the \textsc{BNF} such a data representation 
is (\texttt{Integer} is a primitive data type in Haskell). In this particular 
case, we define a new data type (named \texttt{AE}) with three data constructors: 
\texttt{Num}, \texttt{Add} and \texttt{Sub}. The first constructor expects an 
Integer as argument, while the other two constructors expect two sub-expressions 
of type \texttt{AE}. Let's ignore some details about the \texttt{deriving} directive 
right now, though it explains to the Haskell compiler / interpreter to automatically 
implement support for reading an \texttt{AE} from a string, to show an instance of 
an \texttt{AE} as a string, and to infer the semantics of the \texttt{==} operator 
for the \texttt{AE} data type.

\begin{hscode}\SaveRestoreHook
\column{B}{@{}>{\hspre}l<{\hspost}@{}}%
\column{9}{@{}>{\hspre}l<{\hspost}@{}}%
\column{E}{@{}>{\hspre}l<{\hspost}@{}}%
\>[B]{}\mathbf{data}\;\Conid{AE}\mathrel{=}\Conid{Num}\;\Conid{Integer}{}\<[E]%
\\
\>[B]{}\hsindent{9}{}\<[9]%
\>[9]{}\mid \Conid{Add}\;\Conid{AE}\;\Conid{AE}{}\<[E]%
\\
\>[B]{}\hsindent{9}{}\<[9]%
\>[9]{}\mid \Conid{Sub}\;\Conid{AE}\;\Conid{AE}{}\<[E]%
\\
\>[B]{}\mathbf{deriving}\;(\Conid{Read},\Conid{Show},\Conid{Eq}){}\<[E]%
\ColumnHook
\end{hscode}\resethooks

The \texttt{calc} function we previously specified must be defined to each 
\texttt{AE} construct (or \emph{term}, in some references). Note that this 
is an indutive definition on the constructs of \texttt{AE}. The first definition 
is the base case, and states that \texttt{calc} for a given \texttt{Num n} 
is \texttt{n}. In the other situations, we have to first evaluate 
both \texttt{lhs} (left-hand side) and \texttt{rhs} (right-hand side) before 
calculating the corresponding addition or subtraction. 


\begin{hscode}\SaveRestoreHook
\column{B}{@{}>{\hspre}l<{\hspost}@{}}%
\column{E}{@{}>{\hspre}l<{\hspost}@{}}%
\>[B]{}\Varid{calc}\;(\Conid{Num}\;\Varid{n})\mathrel{=}\Varid{n}{}\<[E]%
\\
\>[B]{}\Varid{calc}\;(\Conid{Add}\;\Varid{lhs}\;\Varid{rhs})\mathrel{=}\Varid{calc}\;\Varid{lhs}\mathbin{+}\Varid{calc}\;\Varid{rhs}{}\<[E]%
\\
\>[B]{}\Varid{calc}\;(\Conid{Sub}\;\Varid{lhs}\;\Varid{rhs})\mathrel{=}\Varid{calc}\;\Varid{lhs}\mathbin{-}\Varid{calc}\;\Varid{rhs}{}\<[E]%
\\[\blanklineskip]%
\>[B]{}\Varid{parse}\mathbin{::}\Conid{String}\to \Conid{AE}{}\<[E]%
\\
\>[B]{}\Varid{parse}\;\Varid{s}\mathrel{=}\Varid{read}\;\Varid{s}{}\<[E]%
\ColumnHook
\end{hscode}\resethooks

Running the test suite helps validate our interpreter (the \texttt{calc} function). 
For instance, if you open the GHCi on the terminal, you can open this module and 
run the test suites, executing in the prompt \texttt{runTestTT tc1} and \texttt{runTestTT tc2}. 
The results must be as follows. 

\begin{mdframed}[style=default]
\noindent \ensuremath{\Conid{Counts}\;\{\mskip1.5mu \Varid{cases}\mathrel{=}\mathrm{1},\Varid{tried}\mathrel{=}\mathrm{1},\Varid{errors}\mathrel{=}\mathrm{0},\Varid{failures}\mathrel{=}\mathrm{0}\mskip1.5mu\}} 

\noindent \ensuremath{\Conid{Counts}\;\{\mskip1.5mu \Varid{cases}\mathrel{=}\mathrm{1},\Varid{tried}\mathrel{=}\mathrm{1},\Varid{errors}\mathrel{=}\mathrm{0},\Varid{failures}\mathrel{=}\mathrm{0}\mskip1.5mu\}} 
\end{mdframed} 

What we have seen is actually quite remarkable, though its full power may not yet 
be apparent. We have shown that a programming language with just the ability to 
represent structured data can represent one of the most interesting forms of 
data, namely programs themselves. That is, we have just written a program that consumes 
programs (in the \texttt{AE} language); perhaps we can even write programs that 
generate programs. The former is the foundation for an interpreter semantics, while the later 
is the foundation for a compiler. This same idea---but with a much more primitive 
language, namely arithmetic, and a much poorer collection of data, namely 
just numbers---is at the heart of the proof of G\"{o}del's Theorem.  

\begin{figure}[t]
\centering
\caption{An illustration of the AST for the simple expression 3 + (10 - 5).}
\label{ast:ae}
{
    \begin{tikzpicture}[->,>=stealth', level/.style={sibling distance = 5cm/#1, level distance = 1.5cm}, scale=0.6,transform shape]
    \node [treenode] {$Add$}
    child
    {
        node [treenode] {$Num$(3)}
    } 
    child
        {
            node [treenode] {$Sub$} 
            child
            {
                node [treenode] {$Num$(10)} 
            }
            child
            {
                node [treenode] {$Num$(5)} 
            }
        }
;
% ------------------------------------------------ put the text into subtree nodes
\end{tikzpicture}
}
\end{figure}


\section{Substitution} 


Even in a simple arithmetic language, we 
sometimes encounter repeated expressions. 
For instance, the Newtonian formula for the 
gravitational force between two objects has a
squared term in the denominator. We would 
like to avoid redundant expressions: they are 
annoying to repeat, we might make a mistake 
while repeating them, and evaluating them 
wastes computational cycles. 

The normal way to avoid redundancy is to introduce 
an identifier.\footnote{As the authors of Concrete Mathematics 
say: ``Name and conquer''.} As its name suggests, 
an identifier names, or identifies, ({\bf the value of}) 
an expression.  We can then use its name in place of 
the larger computation. Identifiers may sound exotic, 
but you are use to them in every programming language 
you have used so far: they are called \emph{variables}. 
We choose not to call them that because the term
``variable'' is semantically 
charged: it implies that the value associated with the identifier 
can change (\emph{vary}). Since our language initially 
won't offer any way of changing the associated value, 
we use the more conservative term ``identifier''. 
For now, they are therefore just names for computed constants. 

Let's first write a few sample programs that use identifiers, 
inventing notation as we go along: 

\begin{tabbing}\tt
~~Let~x~\char61{}~5~\char43{}~5~in~x~\char43{}~x
\end{tabbing}

We want this to evaluate to 20. Here is more elaborate example: 

\begin{tabbing}\tt
~~Let~x~\char61{}~5~\char43{}~5~\\
\tt ~~~in~Let~y~\char61{}~x~\char45{}~3~\\
\tt ~~~~in~y~\char43{}~y~\\
\tt ~\\
\tt ~\char61{}~Let~x~\char61{}~10~in~Let~y~\char61{}~x~\char45{}~3~in~y~\char43{}~y~~~~\char91{}\char43{}~operation\char93{}\\
\tt ~\char61{}~Let~y~\char61{}~10~\char45{}~3~in~y~\char43{}~y~~~~~~~~~~~~~~~~~\char91{}substitution\char93{}~\\
\tt ~\char61{}~Let~y~\char61{}~7~in~y~\char43{}~y~~~~~~~~~~~~~~~~~~~~~~\char91{}\char45{}~operation\char93{}~\\
\tt ~\char61{}~7~\char43{}~7~~~~~~~~~~~~~~~~~~~~~~~~~~~~~~~~~~~\char91{}substitution\char93{}~\\
\tt ~\char61{}~14~~~~~~~~~~~~~~~~~~~~~~~~~~~~~~~~~~~~~~\char91{}\char43{}~operation\char93{}~
\end{tabbing}

En passant, notice that the act of reducing an expression to 
a value requires more than just substitution; 
indeed, it is an interleaving of substitution and calculation 
steps. Furthermore, when we have completed 
substitution we implicitly ``descend''into the inner expression to 
continue calculating. Now, let's define the language more 
formally. To honor the addition of identifiers, we will give 
our language a new name: \lae, short for 
``'Let with arithmetic expressions''. Its \textsc{BNF} is: 

\begin{tabbing}\tt
~~\char60{}LAE\char62{}~\char58{}\char58{}\char61{}~Int~Num~\\
\tt ~~~~~~~~~~\char124{}~Add~\char60{}LAE\char62{}~\char60{}LAE\char62{}\\
\tt ~~~~~~~~~~\char124{}~Sub~\char60{}LAE\char62{}~\char60{}LAE\char62{}~\\
\tt ~~~~~~~~~~\char124{}~Let~\char60{}Id\char62{}~\char60{}LAE\char62{}~\char60{}LAE\char62{}\\
\tt ~~~~~~~~~~\char124{}~Ref~\char60{}Id\char62{}
\end{tabbing}

Notice that we have had to add two rules to the \textsc{BNF}: 
one for associating values with identifiers and 
another for actually using the identifiers. The nonterminal 
\texttt{<Id>} stands for some suitable syntax for identifiers 
(usually a sequence of alphanumeric characters). 

To write programs that process \lae terms, we need a data 
definition to represent those terms. Most 
of \lae carries over unchanged from \ae, but we must pick 
some concrete representation for identifiers. 
Fortunately, Haskell has a primitive type called String, which 
server this role admirably. Nevertheless, it 
is also interesting to introduce a new name 
to the String type, to make clear the purpose of 
identifying expressions. We choose the name \texttt{Id} 
as synonymous to the \texttt{String} data type. Therefore, 
the data definition in Haskell is

\begin{hscode}\SaveRestoreHook
\column{B}{@{}>{\hspre}l<{\hspost}@{}}%
\column{5}{@{}>{\hspre}l<{\hspost}@{}}%
\column{E}{@{}>{\hspre}l<{\hspost}@{}}%
\>[B]{}\mathbf{module}\;\Conid{LAE}\;\mathbf{where}{}\<[E]%
\\[\blanklineskip]%
\>[B]{}\mathbf{import}\;\Conid{\Conid{Test}.HUnit}{}\<[E]%
\\[\blanklineskip]%
\>[B]{}\mathbf{type}\;\Conid{Id}\mathrel{=}\Conid{String}{}\<[E]%
\\
\>[B]{}\mathbf{type}\;\Conid{Value}\mathrel{=}\Conid{Integer}{}\<[E]%
\\[\blanklineskip]%
\>[B]{}\mathbf{data}\;\Conid{LAE}\mathrel{=}\Conid{Num}\;\Conid{Integer}{}\<[E]%
\\
\>[B]{}\hsindent{5}{}\<[5]%
\>[5]{}\mid \Conid{Add}\;\Conid{LAE}\;\Conid{LAE}{}\<[E]%
\\
\>[B]{}\hsindent{5}{}\<[5]%
\>[5]{}\mid \Conid{Sub}\;\Conid{LAE}\;\Conid{LAE}{}\<[E]%
\\
\>[B]{}\hsindent{5}{}\<[5]%
\>[5]{}\mid \Conid{Let}\;\Conid{Id}\;\Conid{LAE}\;\Conid{LAE}{}\<[E]%
\\
\>[B]{}\hsindent{5}{}\<[5]%
\>[5]{}\mid \Conid{Ref}\;\Conid{Id}{}\<[E]%
\\
\>[B]{}\mathbf{deriving}\;(\Conid{Read},\Conid{Show},\Conid{Eq}){}\<[E]%
\ColumnHook
\end{hscode}\resethooks

The \texttt{Let} data constructor expects 
three arguments: the name of the identifier, 
the named expression associated to the identifier, 
and the \texttt{Let} expression body. The \texttt{Ref} 
data constructor expects only one argument: the name 
of the identifier. 

\subsection{Defining Substitution}

Without ceremony, we use the concept of 
\emph{substitution} to explain how the 
\texttt{Let} construct works. We are able to do 
this because substitution is not unique to 
\texttt{Let}: we have studied it for years 
in algebra courses, because that is what 
happens when we pass arguments to 
functions. For instance, let 
$f(x,y) = x^3 + y^3$. Then 

\begin{eqnarray*}
f(12,1) & = & 12^3 + 1^3 = 1728 + 1 = 1729 \\ 
f(10,9) & = & 10^3 + 9^3 = 1000 + 729 = 1729   
\end{eqnarray*} 

Nevertheless, it is a good idea to pin down 
this operation precisely. 

Let's make sure we understand what we are trying 
to define. We want a crisp description of the 
process of substitution, namely what happens when 
we replace an identifier (such as $x$ or \texttt{x}) with 
a value (such as 12 or \texttt{5}) in an expression 
(such as $x^3 + y^3$ or \texttt{x + x}). 

Recall from the sequence of reductions above that 
substitution is a part of, but not the 
same as, calculating an answer for an expression 
that has identifiers. Looking back at the 
sequence of steps in the 
evaluation example above, some of them invoke substitution 
while the rest are calculation as defined for \ae. 
For now, we are first going to pin down substitution. 
Once we have done that, we will revisit the related 
question of calculation. But it will take us a few tries 
to get substitution right! 

\begin{mydef}[Substitution]{def:substitution}
Given an expression like \texttt{Let $x$ = $exp_1$ in $exp_2$},
the components of the \texttt{Let} expression are the 
identifier \texttt{$x$}, the named expression \texttt{$exp_1$}, 
and expression body \texttt{$exp_2$}. To substitute 
the identifier \texttt{$x$} in the expression body \texttt{$exp_1$} 
with the named expression \texttt{$exp_2$}, replace all identifiers 
in the expression body that have the name \texttt{$x$} with the 
named expression (in this case \texttt{$exp_1$}). 
\end{mydef}

Beginning with the program

\texttt{Let x = 5 in x + x}

\noindent 
we will use substitution to replace the identifier 
\texttt{x} with the named expression it is bound 
to (\texttt{5}). The above definition of substitution 
certainly does the trick: after substitution, we 
get 

\texttt{Let x = 5 in 5 + 5} 

\noindent

as we would want. Likewise, it correctly substitutes 
when there are no instances of the identifier. For instance, 

\texttt{Let x = 5 in 10 + 4} 

\noindent the definition of substitution leads 
to \texttt{Let x = 5 in 10 + 4}, since there are 
no instances of x in the expression body. Now consider 

\texttt{Let x = 5 in x + Let x = 3 in 10}

\noindent 

The rules reduce this to \texttt{Let x = 5 in 5 + Let 5 = 3 in 10}. 
Huh? Our substitution rule converted a perfectly reasonable program 
(whose value is 15) into one that 
isn't even syntactically legal, i.e., it would be rejected by a parser 
because the program contains a \texttt{5} where the 
the \bnf tells us to expect an identifier. We definitely don't 
want substitution to have such an effect! It's 
obvious that the substitution algorithm is too naive. To state 
the problem with the algorithm precisely, though, we 
need to introduce a little terminology. 


\begin{mydef}[Binding Instance]{def:bindingInstance}
A binding instance of an identifier is the 
occurrence of the identifier that gives it its 
value. In \lae, the \texttt{Id} position of a 
\texttt{Let} expression is the only binding instance. 
\end{mydef}

\begin{mydef}[Scope]{def:scope}
The scope of a binding instance is the region 
of a program text in which instances of 
the identifier {\bf refer to the value} bound 
by the binding instance. 
\end{mydef}

\begin{mydef}[Bound Instance]{def:boundInstance}
An identifier is bound if it is contained within the 
scope of a binding instance of its name. 
\end{mydef}

\begin{mydef}[Free Instance]{def:freeInstance}
An identifier not contained in the scope of any 
binding instance of its name is said to be 
free. 
\end{mydef}

With this terminology in hand, we can now state 
the problem with the first definition of 
substitution more precisely: it failed to distinguish 
between bound instances (which should be substituted) 
and binding instances (which should not). This leads 
to a refined notion of substitution. 

\begin{mydef}[Substitution, take 2]{def:substitution2}
Given an expression like $Let\ x =\ exp_1\ in\ exp_2$,
the components of the \texttt{Let} expression are the 
identifier \texttt{$x$}, the named expression \texttt{$exp_1$}, 
and expression body \texttt{$exp_2$}. To substitute 
the identifier \texttt{$x$} in the expression body \texttt{$exp_1$} 
with the named expression \texttt{$exp_2$}, replace all identifiers 
in the expression body which are not binding instances and 
that have the name \texttt{$x$} with the 
named expression (in this case \texttt{$exp_1$}). 
\end{mydef}

A quick check reveals that this does not affect the 
outcome of the examples that the previous definition 
substituted correctly. In addition, this definition of 
substitution reduces 
\texttt{Let x = 5 in x + Let x = 3 in 10} to 
\texttt{Let x = 5 in 5 + Let x = 3 in 10}. 

Let's consider a closely related expression 
\texttt{Let x = 5 in x + Let x = 3 in x}. Think 
a little bit. What should the value of this 
expression? Hopefully, we can agree that the 
value of this program is 8 )the left \texttt{x} in the 
addition evaluates to \texttt{5}, the right 
\texttt{x} is given the value \texttt{3}, by the inner 
\texttt{Let}, so the sum is \texttt{8}). The refined substitution 
algorithm, however, converts this expression into 
\texttt{Let x = 5 in 5 + Let x = 3 in 5}, which, 
when evaluated, yields \texttt{10}. 

What went wrong here? Our substitution algorithm 
respected binding instances, but not their 
scope. In the sample expression, the \texttt{Let} introduces 
a new scope for the inner \texttt{x}. The scope of the 
outer \texttt{x} is \emph{shadowed} or \emph{masked} 
by the inner binding. Because substitution doesn't 
recognize this possibility, it incorrectly substitutes 
the inner \texttt{x}. 

\begin{mydef}[Substitution, take 2]{def:substitution2}
Given an expression like $Let\ x =\ exp_1\ in\ exp_2$,
the components of the \texttt{Let} expression are the 
identifier \texttt{$x$}, the named expression \texttt{$exp_1$}, 
and expression body \texttt{$exp_2$}. To substitute 
the identifier \texttt{$x$} in the expression body \texttt{$exp_2$} 
with the named expression \texttt{$exp_1$}, replace all identifiers 
in the expression body which are not binding instances and 
that have the name \texttt{$x$} with the 
named expression (in this case \texttt{$exp_1$}), unless 
the identifier is in a scope different from that 
introduced by \texttt{x}. 
\end{mydef}

While this rule avoids the faulty substitution we have 
discussed earlier, it has the following effect: 
after substitution, the expression \texttt{Let x = 5 in x + Let y = 3 in x} 
becomes \texttt{Let x = 5 in 5 + Let y = 3 in x}. The inner expression 
should result in an error, because \texttt{x} has no value. 
Once again, substitution has changed a correct program into 
an incorrect one! 

Let's understand what went wrong. Why didn't we substitute 
the inner \texttt{x}? Substitution halts at 
the \texttt{Let} because, by definition, every 
\texttt{Let} introduces a new scope, which we said 
should delimit substitution. But this 
\texttt{Let} contains an instance of \texttt{x}, which we 
very much want substituted! So which is it---substitute 
within nested scopes or not? Actually, the two examples 
above should reveal that our latest definition 
for substitution, which might have seemed sensible at first 
blush, is too draconian: it rules out substitution 
within \emph{any} nested scopes. 


\begin{mydef}[Substitution, take 2]{def:substitution2}
Given an expression like $Let\ x =\ exp_1\ in\ exp_2$,
the components of the \texttt{Let} expression are the 
identifier \texttt{$x$}, the named expression \texttt{$exp_1$}, 
and expression body \texttt{$exp_2$}. To substitute 
the identifier \texttt{$x$} in the expression body \texttt{$exp_2$} 
with the named expression \texttt{$exp_1$}, replace all identifiers 
in the expression body which are not binding instances and 
that have the name \texttt{$x$} with the 
named expression (in this case \texttt{$exp_1$}), except within 
\emph{nested scopes of} \texttt{x}. 
\end{mydef}
 
Finally, we have a version of substitution that works. A different, 
more succinct way of phrasing this definition is


\begin{mydef}[Substitution, take 5]{def:substitution2}
Given an expression like $Let\ x =\ exp_1\ in\ exp_2$,
the components of the \texttt{Let} expression are the 
identifier \texttt{$x$}, the named expression \texttt{$exp_1$}, 
and expression body \texttt{$exp_2$}. To substitute 
the identifier \texttt{$x$} in the expression body \texttt{$exp_2$} 
with the named expression \texttt{$exp_1$}, replace all 
free instances of \texttt{x} in the expression body with 
the named expression (in this case, $exp_1$).
\end{mydef}

Recall that we are still defining substitution, not 
evaluation. Substitution is just an algorithm 
defined over expressions, independent of any use in an 
evaluator. It is the calculator's job to invoke substitution 
as many times as necessary to reduce a program down to an 
answer. That is, substitution simply converts 
\texttt{Let x = 5 in x + Let y = 3 in x} into 
\texttt{Let x = 5 in 5 + Let y = 3 in 5}. Reducing this to 
an actual value is the task of the rest of the calculator. Phew! 
Just to be sure we understand this, let's express it in the form 
of a function. 

\begin{hscode}\SaveRestoreHook
\column{B}{@{}>{\hspre}l<{\hspost}@{}}%
\column{E}{@{}>{\hspre}l<{\hspost}@{}}%
\>[B]{}\mbox{\onelinecomment  substitutes the first argument (x) by the second argument (v) }{}\<[E]%
\\
\>[B]{}\mbox{\onelinecomment  in the free occurrences of the let expression body (the third }{}\<[E]%
\\
\>[B]{}\mbox{\onelinecomment  argument of the function). the resulting expression must not have }{}\<[E]%
\\
\>[B]{}\mbox{\onelinecomment  any free occurrence of the first argument.  }{}\<[E]%
\\
\>[B]{}\Varid{subst}\mathbin{::}\Conid{Id}\to \Conid{LAE}\to \Conid{LAE}\to \Conid{LAE}{}\<[E]%
\\
\>[B]{}\Varid{subst}\;\anonymous \;\anonymous \;(\Conid{Num}\;\Varid{n})\mathrel{=}\Conid{Num}\;\Varid{n}{}\<[E]%
\\
\>[B]{}\Varid{subst}\;\Varid{x}\;\Varid{v}\;(\Conid{Add}\;\Varid{lhs}\;\Varid{rhs})\mathrel{=}\Conid{Add}\;(\Varid{subst}\;\Varid{x}\;\Varid{v}\;\Varid{lhs})\;(\Varid{subst}\;\Varid{x}\;\Varid{v}\;\Varid{rhs}){}\<[E]%
\\
\>[B]{}\Varid{subst}\;\Varid{x}\;\Varid{v}\;(\Conid{Sub}\;\Varid{lhs}\;\Varid{rhs})\mathrel{=}\Conid{Sub}\;(\Varid{subst}\;\Varid{x}\;\Varid{v}\;\Varid{lhs})\;(\Varid{subst}\;\Varid{x}\;\Varid{v}\;\Varid{rhs}){}\<[E]%
\\
\>[B]{}\Varid{subst}\;\Varid{x}\;\Varid{v}\;(\Conid{Let}\;\Varid{i}\;\Varid{e1}\;\Varid{e2})\mathrel{=}\bot {}\<[E]%
\\
\>[B]{}\Varid{subst}\;\Varid{x}\;\Varid{v}\;(\Conid{Ref}\;\Varid{i})\mathrel{=}\bot {}\<[E]%
\ColumnHook
\end{hscode}\resethooks
 
The \texttt{subst} function is defined in terms of 
\emph{pattern matching}. In the first case, a substitution 
of any identifier by any named expression within 
a \texttt{Num n} expression body actually returns 
the expression body. When the message body 
is an expression like \texttt{Add e1 e2} or 
\texttt{Sub e1 e2} we return either and \texttt{Add} 
or a \texttt{Sub} expression, respectively, though 
having as sub expressions recursive calls to the 
\texttt{subst} function on their respective 
sub expressions \texttt{e1} and \texttt{e2}. Based on 
the previous definitions, you should implement 
the case for \texttt{subst} on \texttt{Let} expressions. 
This is the most interesting case. Finally, 
substituting a \texttt{Ref i} expression have 
to deal with two new situations. The first, 
we are trying to substitute the identifier \texttt{x} 
by the named expression \texttt{v} within a \texttt{Ref x}. 
In this case, we just return \texttt{v}. In the second, 
we are trying to substitute within a \texttt{Ref i}, where 
\texttt{x != i}, and thus we return \texttt{Ref i}---there 
is no substitution to perform in this case. 

\subsection{Calculating with \texttt{Let}}

We have finally defined substitution, but we still 
have not specified how we will use it to reduce 
expressions to answers. To do this, we must 
modify our calculator. Specifically, we must add 
rules for our two new source language syntactic 
constructs: \texttt{Let} and \texttt{Ref}. 

\begin{itemize}
\item To evaluate \texttt{Let} expressions, we {\bf first calculate} the 
named expression and then substitutes identifier by its value 
in the body of the \texttt{Let} 
expression. 

\item How about identifiers? Well any identifier that is in the scope of 
a \texttt{Let} expression must be replaced with a value when the 
calculator encounters that identifiers binding instance. Consequently, 
the purpose statement of \emph{subst} said there would be no free 
instances of the identifier given as an argument left in the 
result. In other words, \emph{subst} replaces identifiers with 
values before the calculator ever finds them. As a result, any 
\emph{as-yet-unsubstituted} identifier must be free in the whole 
program. The calculator can't assign a value to a free identifier, 
so it halts with an error. 
\end{itemize}

Please, considering the implementation of the \texttt{calc} function 
for \ae, implement a new function (also named \texttt{calc}) for \lae. Consider 
the following test cases. 

\begin{hscode}\SaveRestoreHook
\column{B}{@{}>{\hspre}l<{\hspost}@{}}%
\column{E}{@{}>{\hspre}l<{\hspost}@{}}%
\>[B]{}\Varid{calc}\mathbin{::}\Conid{LAE}\to \Conid{Integer}{}\<[E]%
\\[\blanklineskip]%
\>[B]{}\Varid{calc}\mathrel{=}\bot {}\<[E]%
\\[\blanklineskip]%
\>[B]{}\mbox{\onelinecomment  some HUnit test cases to better understand the calc semantics}{}\<[E]%
\\[\blanklineskip]%
\>[B]{}\Varid{exp1},\Varid{exp2},\Varid{exp3},\Varid{exp4}\mathbin{::}\Conid{String}{}\<[E]%
\\
\>[B]{}\Varid{exp1}\mathrel{=}\text{\tt \char34 Num~5\char34}{}\<[E]%
\\
\>[B]{}\Varid{exp2}\mathrel{=}\text{\tt \char34 Add~(Num~5)~(Num~5)\char34}{}\<[E]%
\\
\>[B]{}\Varid{exp3}\mathrel{=}\text{\tt \char34 Let~\char92 \char34 x\char92 \char34 ~(Add~(Num~5)~(Num~5))~(Add~(Ref~\char92 \char34 x\char92 \char34 )~(Ref~\char92 \char34 x\char92 \char34 ))\char34}{}\<[E]%
\\
\>[B]{}\Varid{exp4}\mathrel{=}\text{\tt \char34 Let~\char92 \char34 x\char92 \char34 ~(Num~5)~(Let~\char92 \char34 y\char92 \char34 ~(Ref~\char92 \char34 x\char92 \char34 )~(Ref~\char92 \char34 y\char92 \char34 ))\char34}{}\<[E]%
\\
\>[B]{}\Varid{exp5}\mathrel{=}\text{\tt \char34 Let~\char92 \char34 x\char92 \char34 ~(Num~5)~(Let~\char92 \char34 x\char92 \char34 ~(Ref~\char92 \char34 x\char92 \char34 )~(Ref~\char92 \char34 x\char92 \char34 ))\char34}{}\<[E]%
\\[\blanklineskip]%
\>[B]{}\Varid{tc01}\mathrel{=}\Conid{TestCase}\;(\Varid{assertEqual}\;\text{\tt \char34 tc01\char34}\;(\Varid{calc}\;(\Varid{parse}\;\Varid{exp1}))\;\mathrm{5}){}\<[E]%
\\
\>[B]{}\Varid{tc02}\mathrel{=}\Conid{TestCase}\;(\Varid{assertEqual}\;\text{\tt \char34 tc02\char34}\;(\Varid{calc}\;(\Varid{parse}\;\Varid{exp2}))\;\mathrm{10}){}\<[E]%
\\
\>[B]{}\Varid{tc03}\mathrel{=}\Conid{TestCase}\;(\Varid{assertEqual}\;\text{\tt \char34 tc03\char34}\;(\Varid{calc}\;(\Varid{parse}\;\Varid{exp3}))\;\mathrm{20}){}\<[E]%
\\
\>[B]{}\Varid{tc04}\mathrel{=}\Conid{TestCase}\;(\Varid{assertEqual}\;\text{\tt \char34 tc04\char34}\;(\Varid{calc}\;(\Varid{parse}\;\Varid{exp4}))\;\mathrm{5}){}\<[E]%
\\
\>[B]{}\Varid{tc05}\mathrel{=}\Conid{TestCase}\;(\Varid{assertEqual}\;\text{\tt \char34 tc05\char34}\;(\Varid{calc}\;(\Varid{parse}\;\Varid{exp5}))\;\mathrm{5}){}\<[E]%
\\[\blanklineskip]%
\>[B]{}\Varid{parse}\mathbin{::}\Conid{String}\to \Conid{LAE}{}\<[E]%
\\
\>[B]{}\Varid{parse}\mathrel{=}\Varid{read}{}\<[E]%
\ColumnHook
\end{hscode}\resethooks
\chapter{Functions and Functions as Values} 

This chapter introduces several concepts 
related to function declaration, the scope 
of functions, and a classification of 
functions. 

\section{An Introduction to Functions}

In the previous chapter, we have 
added identifiers and and the ability 
to name expressions to the language. 
Much of the time, though, simply 
being able to name an expression isn't 
enough: the expression's value 
is going to depend on the context of 
its use. That means the expression needs 
to be parameterized and, thus, it 
must be a \emph{function}. 

Dissecting a \texttt{Let} expression is 
a useful exercise in helping us design 
functions. Consider the program

\begin{tabbing}\tt
~Let~x~\char61{}~5~in~x~\char43{}~3
\end{tabbing}

In this program, the expression \texttt{x + 3} is parameterized 
over the value of \texttt{x}. In that sense, it is just like 
a function definition: in mathematical notation we might 
write: 

\begin{eqnarray*}
f(x) & = & x + 3
\end{eqnarray*}  

Having named and def ind $f$, what do we do with it? The 
\texttt{LAE} program introduces \texttt{x} and than immediately 
binds it to \texttt{5}. The way we bind a function's argument 
to a value is to apply it. Thus, it is as if we wrote: 

\begin{eqnarray*}
f(x) & = & x + 3;\ f(5)
\end{eqnarray*}
 
In general, functions are useful entities to have in programming 
languages, and it would be instructive to model them. 

\subsection{Enriching the Languages with Functions}

To add functions to \texttt{LAE}, we must define their abstract syntax. 
In particular, we must both describe a \emph{function definition} (declaration) and 
provide a means for its \emph{application} or \emph{invocation}. To do the 
latter, we must add a new kind of expression, resulting in the language 
\texttt{F1LAE}. We will presume, as a simplification, that functions consume 
only one argument. This expression language has the following \bnf. 

\begin{tabbing}\tt
~~\char60{}F1LAE\char62{}~\char58{}\char58{}\char61{}~Int~Num~\\
\tt ~~~~~~~~~~\char124{}~Add~\char60{}F1LAE\char62{}~\char60{}F1LAE\char62{}\\
\tt ~~~~~~~~~~\char124{}~Sub~\char60{}F1LAE\char62{}~\char60{}F1LAE\char62{}~\\
\tt ~~~~~~~~~~\char124{}~Let~\char60{}Id\char62{}~\char60{}F1LAE\char62{}~\char60{}F1LAE\char62{}\\
\tt ~~~~~~~~~~\char124{}~Ref~\char60{}Id\char62{}\\
\tt ~~~~~~~~~~\char124{}~App~\char60{}Id\char62{}~\char60{}F1LAE\char62{}~
\end{tabbing}

The expression representing the argument supplied to the 
function is known as the actual parameter. To capture this 
new language, we again have to declare a Haskell 
data type.  

\begin{hscode}\SaveRestoreHook
\column{B}{@{}>{\hspre}l<{\hspost}@{}}%
\column{12}{@{}>{\hspre}l<{\hspost}@{}}%
\column{E}{@{}>{\hspre}l<{\hspost}@{}}%
\>[B]{}\mathbf{module}\;\Conid{F1LAE}\;\mathbf{where}{}\<[E]%
\\[\blanklineskip]%
\>[B]{}\mathbf{import}\;\Conid{\Conid{Test}.HUnit}{}\<[E]%
\\[\blanklineskip]%
\>[B]{}\mathbf{type}\;\Conid{Id}\mathrel{=}\Conid{String}{}\<[E]%
\\
\>[B]{}\mathbf{type}\;\Conid{Name}\mathrel{=}\Conid{String}{}\<[E]%
\\
\>[B]{}\mathbf{type}\;\Conid{FormalArg}\mathrel{=}\Conid{String}{}\<[E]%
\\
\>[B]{}\mathbf{type}\;\Conid{Value}\mathrel{=}\Conid{Integer}{}\<[E]%
\\[\blanklineskip]%
\>[B]{}\mathbf{data}\;\Conid{Exp}\mathrel{=}\Conid{Num}\;\Conid{Integer}{}\<[E]%
\\
\>[B]{}\hsindent{12}{}\<[12]%
\>[12]{}\mid \Conid{Add}\;\Conid{Exp}\;\Conid{Exp}{}\<[E]%
\\
\>[B]{}\hsindent{12}{}\<[12]%
\>[12]{}\mid \Conid{Sub}\;\Conid{Exp}\;\Conid{Exp}{}\<[E]%
\\
\>[B]{}\hsindent{12}{}\<[12]%
\>[12]{}\mid \Conid{Let}\;\Conid{Id}\;\Conid{Exp}\;\Conid{Exp}{}\<[E]%
\\
\>[B]{}\hsindent{12}{}\<[12]%
\>[12]{}\mid \Conid{Ref}\;\Conid{Id}{}\<[E]%
\\
\>[B]{}\hsindent{12}{}\<[12]%
\>[12]{}\mid \Conid{App}\;\Conid{Name}\;\Conid{Exp}{}\<[E]%
\\
\>[B]{}\mathbf{deriving}\;(\Conid{Read},\Conid{Show},\Conid{Eq}){}\<[E]%
\ColumnHook
\end{hscode}\resethooks

Now, let's study function declaration. A function declaration has three 
components: the name of the function, the names of its arguments 
(known as the formal parameters), and the function's body. 
(The function's parameters might have types, which we will 
discuss later in this book). For now, we will presume 
that functions consume only one argument. A simple 
data definition captures this. 

\begin{hscode}\SaveRestoreHook
\column{B}{@{}>{\hspre}l<{\hspost}@{}}%
\column{E}{@{}>{\hspre}l<{\hspost}@{}}%
\>[B]{}\mathbf{data}\;\Conid{FunDec}\mathrel{=}\Conid{FunDec}\;\Conid{Name}\;\Conid{FormalArg}\;\Conid{Exp}{}\<[E]%
\\
\>[B]{}\mathbf{deriving}\;(\Conid{Read},\Conid{Show},\Conid{Eq}){}\<[E]%
\ColumnHook
\end{hscode}\resethooks

Using this definition, one might declare a standard function 
for doubling its argument as: 

\begin{hscode}\SaveRestoreHook
\column{B}{@{}>{\hspre}l<{\hspost}@{}}%
\column{E}{@{}>{\hspre}l<{\hspost}@{}}%
\>[B]{}\Varid{double}\mathbin{::}\Conid{FunDec}{}\<[E]%
\\
\>[B]{}\Varid{double}\mathrel{=}\Conid{FunDec}\;\text{\tt \char34 double\char34}\;\text{\tt \char34 x\char34}\;(\Conid{Add}\;(\Conid{Ref}\;\text{\tt \char34 x\char34})\;(\Conid{Ref}\;\text{\tt \char34 x\char34})){}\<[E]%
\ColumnHook
\end{hscode}\resethooks

Now we are ready to write the calculator, which we will 
call \emph{interp}---short for interpreter-rather than 
\emph{calc} to reflect the fact that our language 
has grown beyond arithmetic. The interpreter must 
consume two arguments: the expression to evaluate 
and the set of known function declarations. Most of 
the rules of \texttt{LAE} remain the same, 
so we can focus on the new rule. 

\begin{hscode}\SaveRestoreHook
\column{B}{@{}>{\hspre}l<{\hspost}@{}}%
\column{E}{@{}>{\hspre}l<{\hspost}@{}}%
\>[B]{}\Varid{interp}\mathbin{::}\Conid{Exp}\to [\mskip1.5mu \Conid{FunDec}\mskip1.5mu]\to \Conid{Value}{}\<[E]%
\\
\>[B]{}\Varid{interp}\mathrel{=}\bot {}\<[E]%
\ColumnHook
\end{hscode}\resethooks

The rule for an application first looks up 
the named function. If this access succeeds, 
then interpretation proceeds in the body 
of the function after first substituting 
its formal parameter with the (interpreted) 
value of the actual parameter. We can 
see the result using GHCi. 

\subsection{The scope of substitution}

Suppose we ask our interpreter to evaluate 
the expression 

\begin{hscode}\SaveRestoreHook
\column{B}{@{}>{\hspre}l<{\hspost}@{}}%
\column{E}{@{}>{\hspre}l<{\hspost}@{}}%
\>[B]{}\Varid{app1}\mathbin{::}\Conid{Exp}{}\<[E]%
\\
\>[B]{}\Varid{app1}\mathrel{=}\Conid{App}\;\text{\tt \char34 f\char34}\;(\Conid{Num}\;\mathrm{10}){}\<[E]%
\ColumnHook
\end{hscode}\resethooks

In the presence of the solitary function definition 

\begin{hscode}\SaveRestoreHook
\column{B}{@{}>{\hspre}l<{\hspost}@{}}%
\column{E}{@{}>{\hspre}l<{\hspost}@{}}%
\>[B]{}\Varid{f}\mathbin{::}\Conid{FunDec}{}\<[E]%
\\
\>[B]{}\Varid{f}\mathrel{=}\Conid{FunDec}\;\text{\tt \char34 f\char34}\;\text{\tt \char34 n\char34}\;(\Conid{App}\;\text{\tt \char34 n\char34}\;(\Conid{Ref}\;\text{\tt \char34 n\char34})){}\<[E]%
\ColumnHook
\end{hscode}\resethooks

What should happen? Should the interpreter try to substitute 
the $n$ in the function position of the application 
with the number $10$, than complains that no such function 
can be found (or even that function lookup fundamentally 
fails because the names of the functions must be identifiers, 
not numbers)? Or should the interpreter decide that function 
names and function arguments live in two different ``spaces'', 
and let the context determines in which space to lookup a name? Languages 
like Scheme take the former approach: the name of a function 
can be bound to a value in a local scope, thereby rendering 
the function inaccessible through that name. This later 
strategy is known as employing namespaces and languages 
like Common Lisp adopt it. 
 
\subsection{The Scope of Function Definitions} 

Suppose our \emph{definition list} contains multiple function 
declarations. How do these interact with one another? 
For instance, suppose we evaluate the following 
input \texttt{eval app2 [g, h]}, where  

\begin{hscode}\SaveRestoreHook
\column{B}{@{}>{\hspre}l<{\hspost}@{}}%
\column{E}{@{}>{\hspre}l<{\hspost}@{}}%
\>[B]{}\Varid{app2}\mathbin{::}\Conid{Exp}{}\<[E]%
\\
\>[B]{}\Varid{app2}\mathrel{=}\Conid{App}\;\text{\tt \char34 f\char34}\;(\Conid{Num}\;\mathrm{5}){}\<[E]%
\\[\blanklineskip]%
\>[B]{}\Varid{g}\mathbin{::}\Conid{FunDec}{}\<[E]%
\\
\>[B]{}\Varid{g}\mathrel{=}\Conid{FunDec}\;\text{\tt \char34 g\char34}\;\text{\tt \char34 n\char34}\;(\Conid{App}\;\text{\tt \char34 h\char34}\;(\Conid{Add}\;(\Conid{Ref}\;\text{\tt \char34 n\char34})\;(\Conid{Num}\;\mathrm{5}))){}\<[E]%
\\[\blanklineskip]%
\>[B]{}\Varid{h}\mathbin{::}\Conid{FunDec}{}\<[E]%
\\
\>[B]{}\Varid{h}\mathrel{=}\Conid{FunDec}\;\text{\tt \char34 h\char34}\;\text{\tt \char34 m\char34}\;(\Conid{Sub}\;(\Conid{Ref}\;\text{\tt \char34 m\char34})\;(\Conid{Num}\;\mathrm{1})){}\<[E]%
\ColumnHook
\end{hscode}\resethooks

What does the mentioned evaluation do? The main expression 
applies $g$ to $5$. The definition of $g$, in turn, invokes 
function $h$. Should $g$ be able to invoke $h$? Should the 
invocation fail because $h$ is defined after $g$ in the list 
of definitions? What if there are multiple bindings 
for a given function's name? We will expect this evaluation 
to reduce to $9$. That is, we employ the more natural interpretation 
that each function can ``see'' every function's definition, 
and the natural assumption that each name is bound at most 
once so we don't need to disambiguate between definitions. 
It is, however, possible to define more sophisticated 
scopes. 

\begin{Exercise}
Implement the \texttt{interp} function as 
specified above. 
\end{Exercise} 

\begin{Exercise}
If a function can invoke every defined function, that 
means it can also invoke itself. This is currently of 
limited value because our \texttt{F1LAE} language lacks 
a harmonious way of terminating recursion. Implement a 
simple conditional construct (\texttt{if0}) which succeeds 
if the term in the first position evaluates to zero, 
and write interesting recursive functions in this language.  
\end{Exercise}  

\section{First Class Functions} 

There is a similarity between a \texttt{Let} expression 
and a function definition applied immediately to a value. 
For instance, not that: 

\begin{tabbing}\tt
~Let~x~\char61{}~5~in~x~\char43{}~3
\end{tabbing}

\noindent is essentially the same as $f(x) = x + 3; f(5)$. Actually, 
that is not quite right: in the math equation, we give the function 
a name ($f$), whereas there is no identifier named $f$ anywhere 
in the \texttt{Let} expression above. We can, however, rewrite 
the mathematical formulation as $f = \lambda x . x + 3; f(5)$, 
which can then be rewritten as $(\lambda x . x + 3)(5)$ to get 
rid of the unnecessary name $f$. Notice, however, that our 
language \texttt{F1LAE} does not permit anonymous functions 
(a concept that currently is also present in 
imperative languages like Java, Scala, and Python, for instance) 
of the style we have used above. Because such a functions 
are useful in their own right, we now extend our study of 
functions. 

\subsection{A Taxonomy of Functions} 

The translation of \texttt{Let} into mathematical 
notation exploits two features of functions: the 
ability to create anonymous functions, and the ability 
to define functions anywhere in the program (in this case, 
in the function position of a Lambda application). Not 
every programming language offers one or both of these 
capabilities. There is, therefore, a taxonomy that 
governs these different features, which we can use 
when discussing what kind of functions a language 
provides. The taxonomy is as what follows.

\begin{description}
\item [first-order] Functions are not values in the language. They 
can only be defined in a designated portion of the program, where 
they must be given names for use in the remainder of the program. The 
functions in \texttt{F1LAE} are of this nature, which explains the 
\texttt{1} in the name of the language. 

\item [higher-order] Functions can return other functions as values. 

\item [first-class] Functions are values with all the rights of 
other values. In particular, they can be supplied as the value 
arguments to functions, returned by functions as answers, and 
stored in data structures. 

\end{description}

\subsection{Enriching \texttt{F1LAE} with First-Class Functions} 

To add \emph{first-class functions} to \texttt{F1LAE}, we must 
proceed as usual, by first defining its concrete and 
abstract syntaxe trees. First, let us examine some concrete programs: 

\begin{tabbing}\tt
~~\char40{}\char92{}x~\char46{}~x~\char43{}~4\char41{}~5~
\end{tabbing}

\noindent This program (consisting of a sole expression) defines a function 
that adds $4$ to its argument and immediately applies this function 
to $5$, resulting in the value $9$. This one

\begin{tabbing}\tt
~~Let~double~\char61{}~\char40{}\char92{}x~\char46{}~x~\char43{}~x\char41{}~\\
\tt ~~~in~\char40{}double~10\char41{}~\char43{}~\char40{}double~5\char41{}
\end{tabbing}

\noindent evaluates to $30$. The program defines a function, 
binds it to \texttt{double}, then uses that name twice in 
slightly different contexts (i.e., it instantiates the formal 
parameter with different actual parameters). From these 
examples, it should be clear that we must introduce two 
new kinds of expressions: anonymous functions and 
anonymous function applications. Here is the revised 
\bnf corresponding to these examples. 

\begin{grammar}
 <Name> ::= <Id>
 
 <Arg> :: = <Id> 

 <FunDec> ::= `def' <Name> <Arg> `=' <FLAE> 
 
 <FLAE> ::= <Num> 
     \alt Add <FLAE> <FLAE>
     \alt Sub <FLAE> <FLAE> 
     \alt Let <Id> <FLAE> <FLAE>
     \alt Ref <Id>
     \alt App <Name> <FLAE> 
     \alt $\lambda$ <Arg> `.' <FLAE>
     \alt AppLambda <FLAE> <FLAE> 
\end{grammar}

In this language, it is possible to declare 
both named functions (using the function declarations) 
and anonymous functions (using the lambda 
abstractions), which might appear anywhere we 
are expecting a \texttt{FLAE} expression (in particular, 
in the first component of a lambda application). Therefore, instead 
of just the name of a function, programmers can write an arbitrary 
expression that must be evaluated to obtain the function 
to apply. The corresponding abstract syntax is:

\begin{hscode}\SaveRestoreHook
\column{B}{@{}>{\hspre}l<{\hspost}@{}}%
\column{11}{@{}>{\hspre}l<{\hspost}@{}}%
\column{E}{@{}>{\hspre}l<{\hspost}@{}}%
\>[B]{}\mathbf{module}\;\Conid{F2LAE}\;\mathbf{where}{}\<[E]%
\\[\blanklineskip]%
\>[B]{}\mathbf{type}\;\Conid{Id}\mathrel{=}\Conid{String}{}\<[E]%
\\
\>[B]{}\mathbf{type}\;\Conid{Name}\mathrel{=}\Conid{String}{}\<[E]%
\\
\>[B]{}\mathbf{type}\;\Conid{FormalArg}\mathrel{=}\Conid{String}{}\<[E]%
\\[\blanklineskip]%
\>[B]{}\mathbf{type}\;\Conid{Value}\mathrel{=}\Conid{Exp}{}\<[E]%
\\[\blanklineskip]%
\>[B]{}\mathbf{data}\;\Conid{FunDec}\mathrel{=}\Conid{FunDec}\;\Conid{Name}\;\Conid{FormalArg}\;\Conid{Exp}{}\<[E]%
\\[\blanklineskip]%
\>[B]{}\mathbf{data}\;\Conid{Exp}\mathrel{=}\Conid{Num}\;\Conid{Integer}{}\<[E]%
\\
\>[B]{}\hsindent{11}{}\<[11]%
\>[11]{}\mid \Conid{Add}\;\Conid{Exp}\;\Conid{Exp}{}\<[E]%
\\
\>[B]{}\hsindent{11}{}\<[11]%
\>[11]{}\mid \Conid{Sub}\;\Conid{Exp}\;\Conid{Exp}{}\<[E]%
\\
\>[B]{}\hsindent{11}{}\<[11]%
\>[11]{}\mid \Conid{Let}\;\Conid{Id}\;\Conid{Exp}\;\Conid{Exp}{}\<[E]%
\\
\>[B]{}\hsindent{11}{}\<[11]%
\>[11]{}\mid \Conid{Ref}\;\Conid{Id}{}\<[E]%
\\
\>[B]{}\hsindent{11}{}\<[11]%
\>[11]{}\mid \Conid{App}\;\Conid{Name}\;\Conid{Exp}{}\<[E]%
\\
\>[B]{}\hsindent{11}{}\<[11]%
\>[11]{}\mid \Conid{Lambda}\;\Conid{Id}\;\Conid{Exp}{}\<[E]%
\\
\>[B]{}\hsindent{11}{}\<[11]%
\>[11]{}\mid \Conid{AppLambda}\;\Conid{Exp}\;\Conid{Exp}{}\<[E]%
\ColumnHook
\end{hscode}\resethooks

To define our interpreter, we must think a little about 
what kinds of values it consumes and produces. Naturally, 
the interpreter consumes values of \texttt{FLAE} expressions 
(and a list of function declarations). What does it 
produces? Clearly, a program that meets \texttt{FLAE} 
must yields numbers. As we have seen above, 
some programs that use functions and applications 
also evaluate to numbers. How about a program 
that consists solely of a function? That is, 
what is the value of the program \texttt{$(\lambda x . x)$}? It 
clearly does not represent a number. It might be 
a function that, when applied to a numeric argument, 
produces a number, but it is not itself a number. We 
instead realized from this that \emph{anonymous functions are also 
values} that may be the result of a computation. 

We could design an elaborate representation for function values, but for 
now, we will remain modest. We will let the function evaluate to its 
abstract syntax representation (i.e., a \texttt{Lambda} structure). For consistency, 
we will also let numbers evaluate to a  \texttt{Num} structure. Thus, the 
result of evaluating \texttt{$(\lambda x . x)$} would be the value \texttt{Lambda ``x'' (Ref ``x'')}.

Now we are ready to write the interpreter. We must pick a type for 
the value that \emph{interp} returns. Since we have decided 
to represent function and number answers using the abstract 
syntax, it makes sense to use \texttt{FLAE} expressions, with 
the caveat that only two kinds of expressions can appear in 
the output: numbers and functions. Our first interpreter will 
use explicit substitution, to offer a direct comparison with 
the interpreters discussed before. 

\begin{hscode}\SaveRestoreHook
\column{B}{@{}>{\hspre}l<{\hspost}@{}}%
\column{E}{@{}>{\hspre}l<{\hspost}@{}}%
\>[B]{}\Varid{interp}\mathbin{::}\Conid{Exp}\to [\mskip1.5mu \Conid{FunDec}\mskip1.5mu]\to \Conid{Value}{}\<[E]%
\\
\>[B]{}\Varid{interp}\mathrel{=}\bot {}\<[E]%
\ColumnHook
\end{hscode}\resethooks

\section{Making Let Expressions Redundant} 

Now that we have functions as first class citizens, we can combine 
lambda abstractions and lambda applications to recover the behaviour 
of \texttt{Let} expressions as a special case. Every time we encounter 
an expression of the form \texttt{Let var = named in body} we can 
replace it with \texttt{$(\lambda var\ .\ body)\ named$} and obtain 
the same effect. The result of this translation reduces some boilerplate 
code that is necessary to interpret the application of lambda and 
let expressions. 

\begin{Exercise}
Implement a pre-processor that performs this translation. 
\end{Exercise} 

\section{Implementing Functions Using Deferred Substitutions} 

Let's examine the process of interpreting the following small 
program. 

\begin{minipage}{.5\textwidth}
\begin{tabbing}\tt
~Let~x~\char61{}~3~\\
\tt ~~in~Let~y~\char61{}~4\\
\tt ~~~in~Let~z~\char61{}~5~\\
\tt ~~~~in~x~\char43{}~y~\char43{}~z
\end{tabbing}
\end{minipage}%
\begin{minipage}{.5\textwidth}
\begin{eqnarray*}
& = & Let\ y = 4\ in\ Let\ z = 5\ in\ 3 + y + z\ (subst) \\ 
& = & Let\ z = 5\ in\ 3 + 4 + z\ (subst) \\
& = & 3 + 4 + 5\ (subst) \\
& = & 12\ (arithmetic)  
\end{eqnarray*}
\end{minipage}

\noindent On the right is the sequence of evaluation steps. To reduce 
it to an arithmetic problem, the interpreter had to apply substitution 
three times: once for each \texttt{Let} expression. This is slow! How slow? 
Well, if the program has size $n$ (measured in abstract syntax tree nodes), 
than each substitution \emph{traverses} the rest of the program once, making 
the complexity of this interpreterter at least $O(n^2)$. That seems rather 
wasteful, surely we can do better. 

How will avoid computational redundancy? We should create and use 
a \emph{repository of deferred substitutions}. Concretly, here 
is the idea. Initially, we have no substitutions to perform, so the 
repository is empty. Every time we encounter a substitution (in the form 
of a \texttt{Let}  or \texttt{Application}), we augment the repository 
with one more entry, recording the identifier's name and the 
value (if \emph{eager}) or expression (if \emph{lazy}) it should
eventually be substituted with. We continue to evaluate without 
actually performing the substitution. 

This strategy breaks a key invariant we had established earlier, which 
is that any identifier the interpreter encounters is of necessity 
free---in the case it had been bound, it would have been replaced 
by substitution. Because we are no longer using substitution, 
we will encounter bound identifiers during interpretation. How do 
will handle them? We must consult the repository in order to 
substitute them. Our new language \textsc{F3LAE} is quite 
similar to the previous one 

\begin{hscode}\SaveRestoreHook
\column{B}{@{}>{\hspre}l<{\hspost}@{}}%
\column{6}{@{}>{\hspre}l<{\hspost}@{}}%
\column{10}{@{}>{\hspre}l<{\hspost}@{}}%
\column{E}{@{}>{\hspre}l<{\hspost}@{}}%
\>[B]{}\mathbf{module}\;\Conid{F3LAE}\;\mathbf{where}{}\<[E]%
\\[\blanklineskip]%
\>[B]{}\mathbf{type}\;\Conid{Name}\mathrel{=}\Conid{String}{}\<[E]%
\\
\>[B]{}\mathbf{type}\;\Conid{FormalArg}\mathrel{=}\Conid{String}{}\<[E]%
\\
\>[B]{}\mathbf{type}\;\Conid{Id}\mathrel{=}\Conid{String}{}\<[E]%
\\[\blanklineskip]%
\>[B]{}\mathbf{data}\;\Conid{FunDec}\mathrel{=}\Conid{FunDec}\;\Conid{Name}\;\Conid{FormalArg}\;\Conid{Exp}{}\<[E]%
\\[\blanklineskip]%
\>[B]{}\mathbf{data}\;\Conid{Exp}\mathrel{=}\Conid{Num}\;\Conid{Integer}{}\<[E]%
\\
\>[B]{}\hsindent{10}{}\<[10]%
\>[10]{}\mid \Conid{Add}\;\Conid{Exp}\;\Conid{Exp}{}\<[E]%
\\
\>[B]{}\hsindent{10}{}\<[10]%
\>[10]{}\mid \Conid{Sub}\;\Conid{Exp}\;\Conid{Exp}{}\<[E]%
\\
\>[B]{}\hsindent{10}{}\<[10]%
\>[10]{}\mid \Conid{Let}\;\Conid{Id}\;\Conid{Exp}\;\Conid{Exp}{}\<[E]%
\\
\>[B]{}\hsindent{10}{}\<[10]%
\>[10]{}\mid \Conid{App}\;\Conid{Name}\;\Conid{Exp}{}\<[E]%
\\
\>[B]{}\hsindent{10}{}\<[10]%
\>[10]{}\mid \Conid{Lambda}\;\Conid{FormalArg}\;\Conid{Exp}{}\<[E]%
\\
\>[B]{}\hsindent{10}{}\<[10]%
\>[10]{}\mid \Conid{LambdaApp}\;\Conid{Exp}\;\Conid{Exp}{}\<[E]%
\\
\>[B]{}\hsindent{6}{}\<[6]%
\>[6]{}\mathbf{deriving}\;(\Conid{Show},\Conid{Eq}){}\<[E]%
\ColumnHook
\end{hscode}\resethooks

\noindent though we have to declare a new type for representing 
our repository of deferred substitutions. 

\begin{hscode}\SaveRestoreHook
\column{B}{@{}>{\hspre}l<{\hspost}@{}}%
\column{E}{@{}>{\hspre}l<{\hspost}@{}}%
\>[B]{}\mathbf{type}\;\Conid{DefrdSub}\mathrel{=}[\mskip1.5mu (\Conid{Id},\Conid{Value})\mskip1.5mu]{}\<[E]%
\ColumnHook
\end{hscode}\resethooks

Several situations must be considered when implementing 
the \texttt{interp} function for \textsc{F3LAE}. For 
instance, consider the following test case

\begin{tabbing}\tt
~Let~x~\char61{}~3~\\
\tt ~~in~Let~f~\char61{}~\char40{}\char92{}y~\char46{}~y~\char43{}~x\char41{}~\\
\tt ~~~~in~Let~x~\char61{}~5~\\
\tt ~~~~~~in~f~4
\end{tabbing}

\noindent Depending on the evaluation strategy and 
scope resolution, its evaluation must result either 
in 7 (static scope) or 9 (dynamic scope). In the later 
case, the value of $x$ within the function definition 
depends on the context of application of $f$, not 
on the scope of its definition.

That is, to properly defer substitution, the value of a 
function should be not only its definition, but also 
the substitutions that were due to be performed on 
it. Therefore, we must define a new datatype for the interpreter's 
return value, which attaches the definition-time repository 
to every function value. Our \texttt{Value} datatype 
is either a \emph{numeric value} or a \emph{closure}, 
a kind of function definition that comes together 
with the list of deferred substitutions that appear 
until its definition. We call this constructed 
value a \emph{closure} because it ``closes'' the 
function body of lambda expressions 
over the substitutions that are waiting to 
occur. 

\begin{hscode}\SaveRestoreHook
\column{B}{@{}>{\hspre}l<{\hspost}@{}}%
\column{12}{@{}>{\hspre}l<{\hspost}@{}}%
\column{E}{@{}>{\hspre}l<{\hspost}@{}}%
\>[B]{}\mathbf{data}\;\Conid{Value}\mathrel{=}\Conid{NumValue}\;\Conid{Integer}{}\<[E]%
\\
\>[B]{}\hsindent{12}{}\<[12]%
\>[12]{}\mid \Conid{Closure}\;\Conid{FormalArg}\;\Conid{Exp}\;\Conid{DefrdSub}{}\<[E]%
\ColumnHook
\end{hscode}\resethooks

When the interpreter encounters a function 
application, it must ensure that the function's 
pending substitutions are not forgotten. It must 
however, ignore the substitutions pending at the 
location of the invocation, for that is 
precisely what led us to dynamic instead of 
static scope. It must instead use the substitutions 
of the invocation location to convert the function 
and argument into values, hope that the 
function expression evaluated to a closure, then 
proceed with evaluating the body of the 
function employing the repository of 
deferred substitutions stored in the closure. 

\begin{Exercise}
Implement the interpreter function for 
F3LAE, considering the following 
specification. 
\end{Exercise} 

\begin{hscode}\SaveRestoreHook
\column{B}{@{}>{\hspre}l<{\hspost}@{}}%
\column{E}{@{}>{\hspre}l<{\hspost}@{}}%
\>[B]{}\Varid{interp}\mathbin{::}\Conid{Exp}\to \Conid{DefrdSub}\to [\mskip1.5mu \Conid{FunDec}\mskip1.5mu]\to \Conid{Value}{}\<[E]%
\\
\>[B]{}\Varid{interp}\mathrel{=}\bot {}\<[E]%
\ColumnHook
\end{hscode}\resethooks

\end{document}
